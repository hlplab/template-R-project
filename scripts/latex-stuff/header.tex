%% coloring of comments by authors
% Define a custom color if needed
\definecolor{authorTFJcolor}{rgb}{0, 1, 1} % Blue
\definecolor{authorXXXcolor}{rgb}{0, 1.0, 0} % Green

% Define commands for each author
\newcommand{\todoTFJ}[1]{\todo[author=TFJ, color=authorTFJcolor]{#1}}
\newcommand{\todoXXX}[1]{\todo[author=XXX, color=authorXXXcolor]{#1}}

\usepackage{fontspec}

%% Pandoc can only set 10, 11, 12 pt
%% uncomment below to set fontsize
%\usepackage[fontsize=13pt]{scrextend}

%\setmainfont{Calibri} % Set main font for latin characters

\setlength{\parskip}{0.15cm} % Set space between paragraphs
%\linespread{1.2}\selectfont % Set line height
%\usepackage[doublespacing]{setspace} % Use double space without changing footnotes line height

%% Special font for IPA
%% Make sure "Doulos SIL" is installed on your computer
%% For other typefaces supporting IPA symbols, see
%% https://en.wikipedia.org/wiki/International_Phonetic_Alphabet#Typefaces
\newfontfamily\ipa{Doulos SIL} % Font for IPA symbols
\DeclareTextFontCommand{\ipatext}{\ipa}



%%%         Section for CJK Characters                   %%%
%%%   You may want to uncomment the code below if        %%%
%%%   you're writing this document with CJK characters   %%%

%\usepackage{xeCJK}  % Uncomment for using CJK characters
%% Set main font for CJK characters
%% Make sure your system has the font set
%\setCJKmainfont[
%	BoldFont={HanWangHeiHeavy}  % Set font for CJK boldface
%    ]{標楷體}    % Set font for normal CJK
%% Some Traditional Chinese fonts: AR PL KaitiM Big5, PingFang TC, Noto Sans CJK TC
%\XeTeXlinebreaklocale "zh"
%\XeTeXlinebreakskip = 0pt plus 1pt

%% Added by Florian to make biblatex and its \refsection work since pandoc does not parse
%% content within latex environments. \refsection in turn is required to allow authors
%% to have multiple separate bibliographies (here: one for main text and one for SI).
%%
%% For some reasons it does seem to be necessary to reload the bibpackage here since
%% ---at least for my system---defining style=apa via:
%%
%%    biblio-style: apa
%%
%% or:
%%
%%    biblatexoptions:
%%      - backend=biber
%%      - maxbibnames=999
%%      - style=apa
%%
%% does not seem to work. Curiously removing the following from the YAML header:
%%
%%     citation_package: biblatex
%%
%% makes knitr (or pandoc?) revert back to the default bibtex treatment, which does
%% not allow multiple bibliographies.
% \usepackage[backend=biber,maxbibnames=999,style=apa]{biblatex}

\newcommand{\brefsection}{\begin{refsection}}
\newcommand{\erefsection}{\end{refsection}}

\newcommand{\changelocaltocdepth}[1]{%
  \addtocontents{toc}{\protect\setcounter{tocdepth}{#1}}%
  \setcounter{tocdepth}{#1}%
}
\setcounter{tocdepth}{-10}
\changelocaltocdepth{-10}
